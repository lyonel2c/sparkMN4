%!TEX root = ../thesis.tex
%*******************************************************************************
%****************************** Third Chapter **********************************
%*******************************************************************************
\chapter{Methodology}

% **************************** Define Graphics Path **************************
\ifpdf
    \graphicspath{{Chapter3/Figs/Raster/}{Chapter3/Figs/PDF/}{Chapter3/Figs/}}
\else
    \graphicspath{{Chapter3/Figs/Vector/}{Chapter3/Figs/}}
\fi

\section{Project phases}
We aim to address all the above mentioned goals by setting out following stages:

\paragraph{Phase 1: Development of novel UGC-specific image recognition models}

Select the proper CNN setup, acquire and preprocess the data and train a set of UGC-specific scene-based and object-based image recognition models, including a complete set of Instagram spam image detection models. Evaluate the performance and scalability of the combined execution of the resulting classifiers and detectors in real-time. Develop strategies that enable improving the performance and scalability of the solution. Evaluate the suitability of the solution when applied to custom models that need to be trained on-demand with small datasets.

\paragraph{Phase 2: Development of an arquitecture to enable complex image recogition pipelines for automatic UGC curation}

Taking as a starting point the setup resulting from Phase 1, it will be developed an efficient and scalable solution to enable chaining multiple classifiers and object detectors in a graph fashion. The resulting architecture will enable the definition of more complex image recogition pipelines for automatic UGC curation. 

\paragraph{Phase 3: Evaluation of the performance and scalability of the obtained solution enabling complex image recogition pipelines in a real scenario}
Define a real scenario and evaluate the performance and scalability of the results of Phase 2 under stress conditions. Examine the impact of different configurations and derive conclusions aiming to pave the way towards systematic and optimized methodologies for automatic UGC curation.


%%%%%%%%%%%%%%%%%%%%%%%%%%%%%%%%%%%%%%%%%%%%%%%%%%%%%%%%%%%%%%%%%%%%%%%%%%%%%%%%%%%%%%%%%%%%%%
%%%%%%%%%%%%%%%%%%%%%%%%%%%%%%%%%%%%%%%%%%%%%%%%%%%%%%%%%%%%%%%%%%%%%%%%%%%%%%%%%%%%%%%%%%%%%%
\section{Associated work plan}

I plan to finish my Ph.D in a total of three years. A tentative work plan of my future activities can be found in Figure \ref{gantt}. 

\begin{landscape}

\begin{figure}[b] 
\centering    
\includegraphics[width=1.0\textwidth]{minion}
\caption[Minion]{Here you can place the GANTT diagram as an image}
\label{gantt}
\end{figure}

\end{landscape}


- J1: This publication will describe the ..TODO.\\
- C2: This publication will evaluate ...TODO...\\
- C3 : This publication will detail...TODO.\\
- J2: This publication will summarize previous contributions on a real scenario...\\


\section{Completed activities}
In this last year, the first six months have been principally spent studying the topic and having theoretical research. Then I implemented ..... 

[DESCRIBE HERE THE WORK WITH SPARK4MN + DL4J. CLARIFY WHEN THE TESTS FINISHED (WHEN YOUR ACCOUNT WAS CLOSED) ]


\section{Resources}

\subsection{Data}

[TODO. Clarify that enven if the public API will change you still have the enough data in a save place]

\subsection{Hardware}

[TODO. Mention Adsmurai machines and also machines from the CAP research group.]
